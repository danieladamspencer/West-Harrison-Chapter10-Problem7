\documentclass{article}

\usepackage{graphicx}
\usepackage{geometry}
\usepackage{amsmath}
\usepackage{enumerate}

\geometry{top=1in,
		left=0.5in,
		right=0.5in,
		bottom=1in}
	
\newcommand{\bone}{\mathbf{1}}
\newcommand{\bphi}{\boldsymbol{\phi}}
\newcommand{\sD}{\mathcal{D}}
\newcommand{\bm}{\mathbf{m}}
\newcommand{\bC}{\mathbf{C}}
\newcommand{\bA}{\mathbf{A}}
\newcommand{\btheta}{\boldsymbol{\theta}}
	
\title{West \& Harrison Chapter 10, Problem 7}
\author{The Dans}
\date{Thursday, March 16, 2017}

\begin{document}
	\maketitle
	
	The quarterly sales and cost index of a confectionary product are given in figure \ref{fig:data}.
	\begin{figure}[h]
		\centering
		\includegraphics[width=0.4\linewidth]{D:/Documents/GitHub/West-Harrison-Chapter10-Problem7/WriteUp/Data}
		\caption{The data associated with the problem}
		\label{fig:data}
	\end{figure}
	The goal is to fit a DLM to the sales with first-order polynomial, regression on cost, and full seasonal components. Based on previous information, 
	\begin{itemize}
		\item the underlying level of the series when cost is zero is 220 with a nominal standard error of 15
		\item the regression coefficient of cost is estimated as -1.5 with a standard error of about 0.7
		\item the seasonal factors for the four quarters of the first year are expected to be -50, 25, 50, -25, with nominal standard errors of 25, 15, 25, and 15, respectively
		\item the trend, regression, and seasonal components are initially assumed to be uncorrelated
		\item the observational variance is estimated as 100, with initial degrees of freedom of 12
	\end{itemize}
	Using this information, we are to analyze the series along the following lines:
	\begin{enumerate}[(a)]
		\item \textbf{Using the information provided for the seasonal factors above, apply Theorem 8.2 to derive the appropriate initial prior that satisfies the zero-sum constraint.}
		
		\textit{Theorem 8.2: Imposing the constraint $\bone'\bphi = 0$ on the prior $(\bphi_0|\sD_0^*) \sim \text{N}(\bm_0^*,\bC_0^*)$ and writing $U = \bone'\bC_0^*\bone$ and $\bA = \bC_0^*\bone/U$ gives the revised joint prior}
		\begin{align*}
			(\bphi_0|\sD_0) & \sim \text{N}(\bm_0,\bC_0),\\
			\bm_0 = \bm_0^* & - \bA\bone'\bm_0^*,\\
			\bC_0 = \bC_0^* & - \bA\bA'U.
		\end{align*} 
		Using the given information for the seasonal components,
		\begin{align*}
			\bm_0^* & = (-50,25,50,-25)' &
			\bC_0^* & = \begin{pmatrix}
				625 & 0 & 0 & 0 \\
				0 & 225 & 0 & 0 \\
				0 & 0 & 625 & 0 \\
				0 & 0 & 0 & 225
			\end{pmatrix}.
		\end{align*}
		Applying Theorem 8.2 yields
		\begin{align*}
			U & = 1700 \\ A & = (0.3676471, 0.1323529, 0.3676471, 0.1323529)' \\
			\bm_0^s & = (-50,  25,  50, -25)' \\
			\bC_0^s & = \begin{pmatrix}
			395.22 & -82.72 & -229.78 & -82.72 \\ 
			-82.72 & 195.22 & -82.72 & -29.78 \\ 
			-229.78 & -82.72 & 395.22 & -82.72 \\ 
			-82.72 & -29.78 & -82.72 & 195.22 \\ 
			\end{pmatrix}.
		\end{align*}
		This leads to the initial prior satisfying the zero-sum constraint for the seasonal components to be $(\bphi_0|\sD_0) \sim \text{N}(\bm_0^s,\bC_0^s)$. Given a state vector $\btheta_t = (\theta_{1,t},\theta_{2,t},\phi_{1,t},\ldots,\phi_{4,t})'$, the initial prior for the DLM is of the form $(\btheta_0|\sD_0) \sim \text{N}(\bm_0,\bC_0)$, where
		\begin{align*}
			\bm_0 & = (220,-1.5, -50,  25,  50, -25)' \\
			\bC_0 & = \begin{pmatrix}
			225.00 & 0.00 & 0.00 & 0.00 & 0.00 & 0.00 \\ 
			0.00 & 0.49 & 0.00 & 0.00 & 0.00 & 0.00 \\ 
			0.00 & 0.00 & 395.22 & -82.72 & -229.78 & -82.72 \\ 
			0.00 & 0.00 & -82.72 & 195.22 & -82.72 & -29.78 \\ 
			0.00 & 0.00 & -229.78 & -82.72 & 395.22 & -82.72 \\ 
			0.00 & 0.00 & -82.72 & -29.78 & -82.72 & 195.22 \\ 
			\end{pmatrix}.
		\end{align*}
		
		\item 
		
	\end{enumerate}
	
	
\end{document}